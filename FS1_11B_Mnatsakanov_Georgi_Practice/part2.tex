% ---------------------------- Problem 1----------------------------------
\subsubsection*{\center Задача № 1.}
{\bf Условие:~}
Дана последовательность $\{a_n\} = \dfrac{1-2n^2}{n^2+3}$ и число $c=-{2}$. Доказать, что 
$$\lim\limits_{n\rightarrow\infty}a_n=c,$$
а именно, для каждого сколь угодно малого числа $\eps>0$ найти наименьшее натуральное число 
$N=N(\eps)$ такое, что $|a_n-c|<\eps$ для всех номеров $n>N(\eps)$.
Заполнить таблицу
\begin{center}
	\begin{tabular}{|c|c|c|c|}
		\hline
		$\eps$ &  $0{,}1$ & $0{,}01$ & $0{,}001$ \\
		\hline
		$N(\eps)$ & & & \\
		\hline
	\end{tabular}
\end{center}
{\bf Решение.~}	
Найдем предел последовательности $a_n = \dfrac{1-2n^2}{n^2+3}$
$$
\lim\limits_{n\rightarrow\infty}\dfrac{1-2n^2}{n^2+3}=\lim\limits_{n\rightarrow\infty}\dfrac{\dfrac{1}{n^2}-2}{1+\dfrac{3}{n^2}}=-{2}
$$
\hspace{1cm}
\vspace{0.5cm}
Что и требовалось доказать.\\
\hspace{0.7cm}
Рассмотрим неравенство |$a_n-c|<\eps,\,\forall\eps>0$, учитывая выражение для $a_n$ и значение $c$ из условия варианта,
получим
$$
\biggl|\frac{1-2n^2}{n^2+3}+2\biggl|<\eps
$$
Неравенство запишем в виде двойного неравентсва и приведём выражение под знаком модуля к общему знаменателю,
получим
$$
-\eps < \frac{7}{n^2+3} < \eps
$$
Заметим, что левое нераенство выполнено для любого номера $n\in\mathbb{N}$, поэтому будем рассматривать только правое неравенство
$$
 \frac{7}{n^2+3} <\eps
$$
Выполнив цепочку преобразований, перепишем неравенство относительно $n^2$, и учитывая, что $n\in\mathbb{N}$, получим
$$
\begin{array}{c}
\dfrac{7}{n^2+3} < \eps,                               \\[8pt]
n^2+3 > \dfrac{7}{\eps}, 						       \\[8pt]
n^2 > \dfrac{7-3\eps}{\eps}                            \\[8pt]
n > \sqrt{\,\dfrac{7-3\eps}{\eps}}, 		            \\[8pt]
N(\eps) = \Biggl[\,\sqrt{\,\dfrac{7-3\eps}{\eps}}\,\Biggr],
\end{array}
$$
где $[\phantom{a}]$ --- целая часть числа. \\
$$
\begin{array}
$N$(0,1) = \Biggl[\,\sqrt{\,\dfrac{7-0,3}{0,1}}\,\Biggr]=8;\\[8pt] \\
$N$(0,01) = \Biggl[\,\sqrt{\,\dfrac{7-0,03}{0,01}}\,\Biggr]=26;\\[8pt]\\
$N$(0,001) = \Biggl[\,\sqrt{\,\dfrac{7-0,003}{0,001}}\,\Biggr]=83;\\[8pt]
\end{array}
$$
Заполним таблицу:
\begin{center}
	\begin{tabular}{|c|c|c|c|}
		\hline
		$\eps$ &  $0{,}1$ & $0{,}01$ & $0{,}001$ \\
		\hline
		$N(\eps)$ & 8 & 26 & 83 \\
		\hline
	\end{tabular}
\end{center}
\textbf{Проверка:}
$$
\begin{array}{l}
|a_9 - c| = \dfrac{7}{84} < 0{,}1,			\\[10pt]
|a_{27} - c| = \dfrac{7}{732} < 0{,}01,	\\[10pt]
|a_{84} - c| = \dfrac{7}{7059} < 0{,}001.
\end{array}
$$

% ---------------------------- Problem 2----------------------------------
\subsubsection*{\center Задача № 2.}
{\bf Условие:~}
Вычислить пределы функций
$$
\begin{array}{cc}
\text{\bf(а):} & \lim\limits_{x\rightarrow3}\dfrac{3x^3-81}{x^2-5x+6}
\\[10pt]
\text{\bf(б):} & \lim\limits_{x\rightarrow\infty}\dfrac{\sqrt{9x^4+3}-\sqrt[3]{x^6+1}}{x^2+100x}},
\\[10pt]
\text{\bf(в):} & \lim\limits_{x\rightarrow-{1}}\dfrac{\sqrt{x+2}-1}{x^3+1},\\[10pt]
\text{\bf(г):} & \lim\limits_{x\rightarrow3}\biggl(\dfrac{6-x}{3}\biggr)^{\tg{\frac{\pi{x}}{6},
\\[10pt]
\text{\bf(д):} & \lim\limits_{x\rightarrow0+}\biggl(\arctg\dfrac{x+1}{x^2+2x}\biggr)^{\frac{\sqrt[3]{x+1}-1}{x},
\\[10pt]
\text{\bf(е):} & \lim\limits_{x\rightarrow2\pi}\dfrac{\sin{7x}-\sin{3x}}{e^{x^2}-e^{4\pi^2}}.
\end{array}
$$
{\bf Решение.~}\\
\text{\bf(а):}
$$
\begin{array}{l}
\lim\limits_{x\rightarrow3}\dfrac{3x^3-81}{x^2-5x+6} = \biggl[\,\dfrac{0}{0}\biggl]\,=
\lim\limits_{x\rightarrow3}\dfrac{3(x-3)(x^2+3x+9)}{(x-3)(x-2)} = 
\lim\limits_{x\rightarrow3}\dfrac{3(x^2+3x+9)}{(x-2)} = 
\dfrac{3\cdot27}{1} = 81.
\end{array}
$$	
\text{\bf(б):}
$$
\begin{array}{l}
\lim\limits_{x\rightarrow\infty}\dfrac{\sqrt{9x^4+3}-\sqrt[3]{x^6+1}}{x^2+100x}} =\biggl[\,\dfrac{\infty-\infty}{\infty}\biggl]\,=
\lim\limits_{x\rightarrow\infty}\dfrac{x^2\biggl(\sqrt{9+\dfrac{3}{x^4}}-\sqrt[3]{1+\dfrac{1}{x^6}}\biggl)}{x^2\biggl(1+\dfrac{100}{x}\biggl)}} = \\ 
=\lim\limits_{x\rightarrow\infty}\dfrac{\sqrt{9+\dfrac{3}{x^4}}-\sqrt[3]{1+\dfrac{1}{x^6}}}{1+\dfrac{100}{x}}} =
\dfrac{3-1}{1}=2
\end{array}
$$	
\text{\bf(в):}
$$
\begin{array}{l}
\lim\limits_{x\rightarrow-{1}}\dfrac{\sqrt{x+2}-1}{x^3+1} =\biggl[\,\dfrac{0}{0}\biggl]\,=
\Biggl|
\begin{array}{l}
t = x + 1 \\ t\rightarrow0 \\ x=t-1
\end{array}
\Biggl| = 
\lim\limits_{t\rightarrow0}\dfrac{\sqrt{t+1}-1}{(t-1)^3+1}  = 
\biggl|
\begin{array}{l}
(1-t)^{\frac12} \sim \frac12t 
\end{array}
\biggr| = \\
=\lim\limits_{t\rightarrow0}\dfrac{\frac12t}{(t^3-3t^2+3t-1+1}=\lim\limits_{t\rightarrow0}\dfrac{\frac12t}{t(t^2-3t+3)}=\lim\limits_{t\rightarrow0}\dfrac{1}{2(t^2-3t+3)}=\dfrac{1}{2\cdot3}=\dfrac16
\end{array}
$$               
\text{\bf(г):}	
$$
\begin{array}{l}
\lim\limits_{x\rightarrow3}\biggl(\dfrac{6-x}{3}\biggr)^{\tg{\frac{\pi{x}}{6}}} = \biggl[\,{1}^{\infty}\biggl]\,=
\lim\limits_{x\rightarrow3}{e^\biggl\ln{\biggl(\dfrac{6-x}{3}\biggr)^{\tg{\frac{\pi{x}}{6}}}}}=
\lim\limits_{x\rightarrow3}{e^{\tg{\frac{\pi{x}}{6}}\cdot\ln{(\frac{6-x}{3})}}}=
\Biggl|
\begin{array}{l}
t = x - 3 \\ t\rightarrow0\\ x=t+3
\end{array}
\Biggl| =\\ \\
\lim\limits_{t\rightarrow0}{\biggl{e}^{\biggl-\ctg{\dfrac{\pi{t}}{6}}\cdot\biggl\ln{\dfrac{3-t}{3}}}}=
\biggl{e}^{\lim\limits_{t\rightarrow0}{\biggl{-\ctg}{\dfrac{\pi{t}}{6}}\cdot\biggl\ln{\biggl1-\dfrac{t}{3}}}}=
\Biggl|
\begin{array}{l}
\biggl\ln{\biggl(\biggl1-\dfrac{t}{3}\biggl)} \sim -\dfrac{t}{3}\\
\biggl\ctg}{\dfrac{\pi{t}}{6}}=\dfrac{1}{\biggl{\tg}{\dfrac{\pi{t}}{6}}} \sim \dfrac{1}{\dfrac{\pi{t}}{6}}}=\dfrac{6}{\pi{t}}}}
\end{array}
\Biggl| =\\
\biggl{e}^{\lim\limits_{t\rightarrow0}{-\dfrac{6}{\pi{t}}\cdot\biggl(-\dfrac{t}{3}\biggl)}}=\Biggl{e}^{\dfrac{2}{\pi}}}
\end{array}
$$
\text{\bf(д):}
$$
\begin{array}{l}
\lim\limits_{x\rightarrow0+}\biggl(\arctg\dfrac{x+1}{x^2+2x}\biggr)^{\frac{\sqrt[3]{x+1}-1}{x}} = \lim\limits_{x\rightarrow0+}\biggl(\arctg\dfrac{1+\dfrac{1}{x}}{x+2}\biggr)^{\frac{\sqrt[3]{x+1}-1}{x}}=
\biggl(\dfrac{\pi}{2}\biggr)^{\lim\limits_{x\rightarrow0+}{\frac{\sqrt[3]{x+1}-1}{x}}} =
\Biggl|
\begin{array}{l}
{\frac{\sqrt[3]{x+1}-1}{x}} \sim \dfrac{1}{3}x
\end{array}
\Biggl| = \\
=\biggl(\dfrac{\pi}{2}\biggr)^{\lim\limits_{x\rightarrow0+}{\dfrac{\frac{1}{3}x}{x}}}=
\sqrt[3]{\,\dfrac{\pi}{2}}.
\end{array}
$$
\text{\bf(е):}
$$
\begin{array}{l}
\lim\limits_{x\rightarrow2\pi}\dfrac{\sin{7x}-\sin{3x}}{e^{x^2}-e^{4\pi^2}} = 
\Biggl|
\begin{array}{ll}
t = x - 2\pi 	\\ 
t\rightarrow0 \\
x = t + 2\pi \\
\end{array}
\Biggr| =
\lim\limits_{t\rightarrow0}\dfrac{\sin{(7(t + 2\pi))}-\sin{(3(t + 2\pi))}}{e^{(t + 2\pi)^2}-e^{4\pi^2}} = \\ \\
=\lim\limits_{t\rightarrow0}\dfrac{\sin{7t}-\sin{3t}}{e^{4\pi^2}\cdot (e^{t^2+4t\pi}-1)}=\biggl[\,\dfrac{0}{0}\biggl]\,
=\lim\limits_{t\rightarrow0}\dfrac{2\cdot \sin{2t} \cdot \cos{5t}}{e^{4\pi^2}\cdot (e^{t^2+4t\pi}-1)}
= \biggl|
\begin{array}{l}
e^{t^2+4t\pi}-1 \sim t^2+4t\pi 	\\ \\
\sin(2t) \sim 2t
\end{array}
\biggl| =\\ \\
=\lim\limits_{t\rightarrow0}\dfrac{2\cdot {2t} }{e^{4\pi^2}\cdot t\cdot{(t+4\pi)}}
=\dfrac{2 \cdot 2}{e^{4\pi^2} \cdot  4\pi} = \dfrac{1}\pi \cdot {e^{4\pi^2}}
\end{array}
$$ \\ \\
\text{\bfОтвет:}
$$
\begin{array}{c  c  c  c  c  c}
\text{\bf{а) }}81{;}&
\text{\bf{б) }}{2}{;}&
\text{\bf{в) }}\dfrac{1}{6}{;}&
\text{\bf{в) }}\biggl{e}^{\dfrac{2}{\pi}}}{;}&
\text{\bf{д) }}\sqrt[3]{\,\dfrac{\pi}{2}}{;}&
\text{\bf{е) }} \dfrac{1}\pi \cdot {e^{4\pi^2}}.
\end{array}
$$
% ---------------------------- Problem 3----------------------------------
\subsubsection*{\center Задача № 3.}
{\bf Условие:~}\\
\text{\bf(а):} Показать, что данные функции
$f(x)$ и $g(x)$ являются бесконечно малыми или бесконечно большими
при указанном стремлении аргумента. \\
\text{\bf(б):} Для каждой функции $f(x)$ и $g(x)$ записать главную часть
(эквивалентную ей функцию)  вида $C(x-x_0)^{\alpha}$ при $x\rightarrow x_0$ или $Cx^{\alpha}$
при $x\rightarrow\infty$, указать их порядки малости (роста). \\
\text{\bf(в):} Сравнить функции $f(x)$ и $g(x)$ при указанном стремлении.
\begin{center}
	\begin{tabular}{|c|c|c|}
		\hline
		№ варианта & функции $f(x)$ и $g(x)$ & стремление \\[6pt]
		\hline
		13 & $f(x) =\sqrt{x+\sqrt{x}}} ,~g(x)=\dfrac{\sqrt[3]{x}}{e^{\frac{1}{\sqrt{x}}}-1}$ & $x\rightarrow{+}\infty$ \\
		\hline
	\end{tabular}
\end{center}
{\bf Решение.~}\\
\text{\bf(а):}~Покажем, что $f(x)$ и $g(x)$ бесконечно большие функции,
$$
\begin{array}{cc}
\lim\limits_{x\rightarrow{+}\infty}f(x) = \lim\limits_{x\rightarrow\infty}\sqrt{x+\sqrt{x}}} = \infty \implies f(x) - \text{Б/б при }  x\rightarrow{+}\infty \\ \\
\lim\limits_{x\rightarrow{+}\infty}g(x) = \lim\limits_{x\rightarrow{+}\infty}\dfrac{\sqrt[3]{x}}{e^{\frac{1}{\sqrt{x}}}-1} = 
\biggl|
e^{\frac{1}{\sqrt{x}}}-1 \sim \frac{1}{\sqrt{x}}
\biggl|=
 \lim\limits_{x\rightarrow{+}\infty}\dfrac{\sqrt[3]{x}}{\frac{1}{\sqrt{x}}}=
\lim\limits_{x\rightarrow{+}\infty}{\sqrt[6]{x^5}}  = \infty \implies \\ \\\implies g(x) - \text{Б/б при }  x\rightarrow{+}\infty
\end{array}
$$	
\text{\bf(б):}~Так как $f(x)$ и $g(x)$ бесконечно большие функции, то эквивалентными им будут функции вида 
$Cx^{\alpha}$ при $x\rightarrow{+}\infty$. 
\\Найдём эквивалентную для $f(x)$ из условия
$$
\lim\limits_{x\rightarrow{+}\infty}\dfrac{f(x)}{x^{\alpha}} = C,
$$
где $C$ --- некоторая константа. Рассмотрим предел
$$
\lim\limits_{x\rightarrow{+}\infty}\dfrac{f(x)}{x^{\alpha}} = \\
\lim\limits_{x\rightarrow{+}\infty}\dfrac{\sqrt{x+\sqrt{x}}}{x^{\alpha}} =
\lim\limits_{x\rightarrow{+}\infty}\dfrac{\sqrt{x}\cdot\sqrt{1+\dfrac{1}{\sqrt{x}}}}{x^{\alpha}}.
$$
При $\alpha=\dfrac{1}{2}$ последний предел равен $1$, отсюда $C=1$ и 
$$
f(x)\sim \sqrt{x}~\text{при}~ x\rightarrow{+}\infty.
$$
\begin{center}
{Порядок роста функции $f(x)$  равен $ {\dfrac{1}{2}}$ }\\
\end{center}
Аналогично, рассмотрим предел
$$
\lim\limits_{x\rightarrow{+}\infty}\dfrac{g(x)}{x^{\alpha}} = 
\lim\limits_{x\rightarrowv{+}\infty}\dfrac{\sqrt[3]{x}}{x^{\alpha}(e^{\frac{1}{\sqrt{x}}}-1)}=
\biggl|
e^{\frac{1}{\sqrt{x}}}-1 \sim \frac{1}{\sqrt{x}}
\biggl|=
\lim\limits_{x\rightarrow{+}\infty}\dfrac{\sqrt[3]{x}}{x^{\alpha}\cdot{\dfrac{1}{\sqrt[3]{x}}}} =
\lim\limits_{x\rightarrow{+}\infty}\dfrac{\sqrt[6]{x^5}}{x^{\alpha}}.
$$
При $\alpha=\dfrac{5}{6}$ последний предел равен $1$, отсюда $C=1$ и
$$
g(x)\sim \sqrt[6]{x^5}~\text{при}~x\rightarrow{+}\infty.
$$
\begin{center}
{Порядок роста функции $g(x)$  равен $ {\dfrac{5}{6}}$ }\\
\end{center}
\text{\bf(в):}~Для сравнения функций $f(x)$ и $g(x)$ рассмотрим предел их отношения при указанном стремлении
$$
\lim\limits_{x\rightarrow{+}\infty}\dfrac{f(x)}{g(x)}.
$$
Заменим  исходные функции на эквивалентные, определенные в пункте (б), получим
$$
\lim\limits_{x\rightarrow{+}\infty}\dfrac{f(x)}{g(x)} = 
\lim\limits_{x\rightarrow{+}\infty}\dfrac{\sqrt{x}}{\sqrt[6]{x^5}} = 
\lim\limits_{x\rightarro{+}\infty} \dfrac{1}{\sqrt[3]{x}} = 0.  
$$
Отсюда, $f(x)$ =$o$($g(x)$), то есть функия $f(x)$ бесконечно малая относительно функции $g(x)$ при  $x\rightarrow{+}\infty.